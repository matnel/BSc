\documentclass[finnish,gradu,twoside,12pt]{tktltiki}

\usepackage[utf8]{inputenc}
\usepackage{natbib}


\usepackage{graphicx}
\usepackage{hyperref}
\usepackage{rotating}
\usepackage{float}
\usepackage{microtype}

\usepackage{algpseudocode}
\usepackage{algorithm}

\usepackage{listings}


\usepackage{caption}
\usepackage{subcaption}

\usepackage{todonotes}

\begin{document}

%% \onehalfspacing
%\doublespacing

\title{Laskennallinen yhteiskuntatiede}
\author{Matti Nelimarkka}
\date{5.5.2011}
\level{LuK --tutkielma}

\numberofpagesinformation{}

\keywords{}

\maketitle

\begin{abstract}
Työssä esitellään laskennallisten menetelmien vaikutusta yhteiskuntatieteelliseen tutkimukseen kolmantena laajana menetelmällisenä perheenä laadullisen ja määrällisen tutkimuksen lisäksi. Laskennallinen yhteiskuntatiede (\textit{computational social science}) soveltaa moderneja laskennallisia menetelmiä, kuten kone-oppimista ja tilastollista analyysiä, yhteiskuntaa koskevien kysymysten esittämiseen ja tulkintaan.

Erityisesti työssä keskitytään n menetelmään, lista. Menetelmät on valittu esittämään tällä hetkellä käytössä olevia laskennallisen yhteiskuntatieteen sovelluskohteita. Kunkin menetemän kohdalta esitetään lyhyesti sen laskennallinen teoria sekä tutkimuksia, missä kyseistä menetelmää on sovellettu.

Yksittäisten tutkimusten tarkastelun kautta voidaan nostaa esille suurempia teemakokonaisuuksia laskennallisen yhteiskuntatieteen kehityksen kannalta. Ensimmäisenä käsittelen laskennallisen yhteiskuntatieteen validiteettiä ja reliabiliteettia, eli tulosten oikeellisuutta ja uskottavuutta. Toinen merkittävä teema on monitieteellinen työskentely, minkä koen välttämättömänä laskennallisen yhteiskuntatieteen alalla, mutta tunnustan siihen liittyvät ongelmat. Viimeisenä keskustelun osiona pohdin, voiko laskennallinen tiede edustaa kuhnilaisessa mielessä paradigman muutosta, millä olisi laajempi vaikutus yhteiskuntatieteen kehitykseen ja tulevaisuuteen.
\end{abstract}

\mytableofcontents

\section{Johdanto}

Laskennallisten menetelmien käyttö tutkimusmetelmänä muiden tieteenalojen osana on merkittävä sovellusalue tietojenkäsittelytieteille. Esimerkiksi bioinformatiikka sekä laskennallinen biologia keskittyvät aineiston analyysiin tarvittavien algoritmien ja tiedonlouhinnan kehitykseen. Vastaavasti laskennallisessa kielitieteessä pyritään löytämään laajoista tekstimassoista \textit{(korpuksista)} säännönmukaisuuksia sekä kehittää kuluttajille hyödyllisiä sovelluksia automaattisen kääntämisen piirissä. Opetusalalla tiedonlouhinta mahdollistaa kehittyneempien vuorovaikutteisten sovellusten kehittämisen, kun tiedonlouhinnan avulla mallinnetaan opiskelijan oppimista ja sen muutosta, ja tämän perusteella muutetaan sovelluksien toimintaa. Myös yhteiskuntatieteiden piirissä on herännyt mielenkiintoa soveltaa tietojenkäsittelytieteen menetelmiä ja osaamista tutkimuksen osana, esimerkiksi aineiston käsittelyssä, säännönmukaisuuksien etsimisessä ja yksilön sekä ryhmien toiminnan mallintamisessa.

Viimeaikoina tutkimuskentällä -- varsinkin tietojenkäsittelytieteilijöiden osalta -- on noussut esille suurien tietomassojen käsittely laskennallisesti \textit{(big data analysis)}. Erityisesti viimeaikaisen innostuksen taustalla on niin Internetin kautta \citep{adamic05,notess02} kuin elinympäristöstä kerättyjen aineistojen \citep{eagle06,oulasvirta12} kerääminen sekä tarjoaminen muille tutkijoille. Esimerkiksi sosiologian ja sosiaalipsykologian tutkimuksessa mahdollisuus seurata ihmisten välistä viestintää, sijaintia ja muita tietoja \textit{(reality mining)} mahdollistaa esimerkiksi ystävyyssuhteiden tarkastelun uusilla tavoilla \cite{Karikoski2011a,Nelimarkka2012}. Kuitenkin suurien tietomassojen perusteella tehtävää tutkimusta kohtaan on nostettu esille myös kritiikkiä. Esimerkiksi \citet{Boyd2012a} huomauttavat, että suuret tietomassojen laatu ja hyödyllisyys riippuvat aineistossa käytössä olevista mittareista sekä ympäristön ja taustalla olevien ilmiöiden ymmärtämisestä.

Merkittävää onkin huomata, että laskennallisen yhteiskuntatieteen sovellukset ovat laajempia kuin vain suurten tietomassojen keräämisessä ja käsittelyssä. Kuitenkin, yhteiskuntatieteissä käsitellään monia erilaisia tutkimuskysymyksiä, joihin voidaan vastata laskennallisin menetelmin. Ennen tarkempaa paneutumista laskennallisuuteen ja tietojenkäsittelytieteeseen on kuitenkin välttämätöntä esittää yhteiskuntatieteiden oppihistoriaa ja nykyisin vakiintuneita menetelmiä lyhyesti.

%% Tällöin perinteiset ei-laskennalliset menetelmät eivät välttämättä sovellu tutkimukseen, ja samaan aikaan laajat aineistot mahdollistavat aiemmin haastavina pidettyjen menetelmien käytön -- esimerkiksi verkosto-analyysin ongelma on ollut aineiston vaikea keräys \citep{a}. Tällöin myös rahoittajien korostama poikkitieteellinen lähestymistapa \citep[60]{tieteentila12} on ensinnäkin mahdollinen resurssien myötä ja toisaalta välttämätön, koska perinteiset lähestymistavat eivät käytä hyödykseen mahdollisuuksia joita laskennallisten tieteiden kehitys on luonut.

\subsection{Yhteiskuntatieteen oppihistoriaa}

Työssä esitetään laskennallisia menetelmiä joita sovelletaan yhteiskuntaa koskeviin kysymyksiin vastaamiseksi: laskennallisen yhteiskuntatieteen malleihin, sovelluksiin sekä tieteenfilosofiaan. Työssä yhteiskuntatiede nähdään laajasti yhteiskuntaa tutkivien tieteiden joukkona, joihin kuuluvat esimerkiksi politiikan tutkimus, taloustiede, sosiologia, sosiaalipyskologia ja viestinnän tutkimus\footnote{Tässä työssä esimerkit on erityisesti valittu politiikan tukimuksen näkökulmasta, mutta esitetyt menetelmät ovat laajemmin sovellettavissa laajemminkin yhteiskuntatieteisiin.}. Tutkimusaloille tyypillistä on samankaltaisten tutkimuskysymysten lähestyminen kunkin alan oman oppihistorian sekä käsitteistön kautta \citep{a}. Esimerkiksi poliitikkojen sosiaalisen median käyttöä voidaan lähestyä niin politiikan tutkimuksen, viestinnän tutkimuksen kuin esimerkiksi sosiologian näkökulmasta. Esimerkki.

Kuten yllä olevista esimerkeistä huomataan, eri yhteiskuntatieteet käsittelevät samankaltaisia kysymyksiä, kuitenkin erilaisin painotuksin. Nykyinen eriytyminen eri tieteenaloihin on osa tieteenalan laajempaa kehittymistä nykyiseen käsitykseen yhteiskuntatieteestä. Jotta yhteiskuntatieteen nykyinen tila, sen puutteet ja laskennallisten menetelmien tarjoamat mahdollisuudet ymmärrettäisiin, niin on tarpeen esitellä yhteiskuntatieteen oppihistoriaa tiiviisti. Eräs merkittävä muutos joka valtavirran tutkimuksessa sijoittuu 1950 -- 1960-luvuille on behavioralismin vaikutus yhteiskuntatieteisiin. Ennen tätä muutosta yhteiskuntatieteet, varsinkin 1800-luvun lopussa sekä 1900-luvun alussa, olivat kuvailevia ja normatiivisia: erilaisia havaintoja kuvailtiin varsin laajasti \citep{x} ja yhteiskuntafilosofia\footnote{Jonka historia toki on merkittävästi laajempi, esimerkiksi Platonin Valtio voidaan nähdä yhteiskuntaa kuvaavana filosofiana ja Hobbesin sekä Locken sopimusteoriamallien kautta muodostetut käsitteellistykset valtiosta toimijana ovat hyviä esimerkkejä perinteisestä normatiivisesta ajattelusta} oli yhteiskuntatieteen keskeistä sisältöä. Kuitenkin, behavioralismin myötä käänne empiiriseen tutkimukseen, todellisten ilmiöiden havainnointiin ja niiden arviointiin, oli merkittävä. Tätä siirtymää kuvaa \citep[yy]{x} kuvaus yhteiskuntatieteestä:

TODO

Samaan aikaan behavioralismi kuitenkin loi odotuksia käytetyistä lähestymistavoista, behavioralistit kokivat, että luonnontieteiden lähestymistapaa tulisi soveltaa yhteiskuntatieteissä. Tämä nosti esille vastaliikkeen, kriittisen koulukunnan. Kriittisen koulukunnan mukaan yhteiskunnan tutkimuksessa on tarpeen myös luonnontieteistä poikkeava käsittelytapa, missä esimerkiksi tutkimushypoteesien rooli ei ole yhtä keskeinen. Samoin kriittinen koulukunta kyseenalaisti behavioralismin objektiivisuuden, argumentoiden, että tutkimuksen operationalisoinnit luovat jo subjektiivisen ulottuvuuden ja tulosten tulkinta on edelleen subjektiivista. Tietenkin, myös kriittisen koulukunnan työskentelyä kuvaa subjektiivisuuden keskeinen luonne tutkimuksen osana -- mutta, tämä subjektiivisuus onkin usein osa kaikkea yhteiskuntatiedettä.

Modernissa empiirisessä yhteiskuntatieteesä voidaan siis nostaa esille kahden laajemman lähestymistavan soveltaminen tutkimuskysymyksiin. Tutkimuksessa sovelletaan yleensä laadullista tai määrällistä tutkimusotetta, missä ensimmäisessä pääpaino on enemmän kuvailevassa aineistossa ja sen tulkinnassa, jälkimmäinen taas pyrkii mitattaviin aineistoihin ja selittämiseen matemaattisia menetelmiä hyödyntäen \citep{a,b}. Sellaisenaan voimakasta menetelmäjakoa on kritisoitu, koska useiden ilmiöiden kohdalla olisi mielekästä soveltaa molempia tutkimusmenetelmiä: laadullisen tutkimuksen perinteinen haaste on tutkimuksen yleistettävyys koko joukkoon, kun taas määrällisen tutkimuksen tapauksessa kausaalisen tekijän keskeinen luonne voi helposti jäädä selvittämättä \citep{a,b,c}.

Kritiikkini kausaalisuuden puutteesta voi olla yllättävä, joten valaistaan tätä esimerkillä poliitikoiden verkkomedian käytön tutkimuksella. On havaittu, että poliitikot eivät suosi vuorovaikutuksellista ja kaksisuuntaista verkkomedian käyttöä, vaan käyttävät verkkomediaa perinteisen median jatkona \citep{Golbeck2010,}. Kuitenkaan, nämä määrällistä lähestymistapaa soveltavat tutkimukset eivät täysin pysty selittämään miksi näin on. Tällöin on tarve laadulliselle tutkimukselle, kuten \citet{Stromer-Galley2000} havannointityölle ehdokkainen parissa, missä hän esittää, että ehdokkaat pelkäävät interaktiivisen viestinnän altistavan heidät kyseenalaistamiselle ja mahdollisesti sotkevan kampanjaviestintää. Näin vakuuttavaan yhteiskuntatieteelliseen argumentaatioon vaaditaan toisaalta määrällistä, koko tutkittavan joukon kattavaa havaintoa sekä laadullista työskentelyä, mikä joko selvittää havaintoja -- kuten yllä -- tai nostaa esille uusia tutkimuskysymyksiä.

\subsection{Laskennallinen yhteiskuntatiede}

Määrällisestä lähestymistapaa edustavat tilastollisten menetelmien soveltaminen aineiston käsittelyssä, mutta myös formaalit menetelmät, kuten peliteoria voidaan laskea osaksi tätä lähestymistapaa \citep{a}. Myös tarkastelun kohteena oleva laskennallinen yhteiskuntatiede (\textit{computational social science}) edustaa määrällistä lähestymistapaa. Laskennalliset tieteet määritellään \citep{a,b,c}. Vastaava esitys on myös nähtävissä laskennallisen yhteiskuntatieteen määritelmissä:

\begin{description}
\item[\citet{cioffi-revilla10}] määrittelee laskennallisen yhteiskuntatieteen laajasti seuraavien laskennallisten menetelmien soveltamisena: tiedon uuttamisen menetelmät (\textit{information extraction}), sosiaalisten verkostojen analyysi, paikkatietojärjestelmien käyttämisen ja mallinnuksen sekä simulaation menetelmät. Hän myös arvioi, että tiedon visualisointimenetelmät voivat myöhemmin laajentua käytettäväksi laskennallisen yhteiskuntatieteen menetelmänä. Hän siis näkee, että laskennallista tiedettä voidaan määritellä tiettyjen menetelmien käyttönä.
\item[\citet{lazer09}] taas esittää laskennallisen yhteiskuntatieteen laajojen aineistomäärien keräämisenä ja käsitttelynä määrällisten menetelmien avulla. He nostavat esille, että laskennallinen yhteiskuntatiede on aineistolähtöistä (\textit{data-driven}), minkä tulkitsen tarkoittavan, että perinteinen hypoteeseihin ja niiden tarkasteluun perustuvat tilastolliset menetelmät eivät olisi laskennallista yhteiskuntatiedettä.
\item[\cite{bankes02}] kuvaavat laskennallisia epistemologioita (\textit{computational epistemology}), eli käsityksiä hyvästä tieteestä. Toisaalta he ehdottavat, että laskennallinen yhteiskuntatieteen tarkoitus on luoda selittäviä malleja, joiden avulla yhteisöä voidaan tutkia. Toisaalta, he nostavat esille myös mahdollisuuden käyttää näitä malleja ei vain yhteisön tutkimiseen vaan myös laskennallisten koeasetelmien suorittamiseen. Tässä määritelmässä siis tärkeintä on tutkittavan ilmiön mallinnus laskennallisin menetelmin.
\end{description}

Kuten havaitaan, tutkimusalan määritelmät eivät ole kaikilla tutkijoilla samankaltaisia, johtuen myös erilaista lähestymistavoista laskennallisen yhteiskuntatieteen määritelmään. Kuten \citet{cioffi-revilla10} huomauttaa, laskennallisen tieteen osalta voidaan erottaa laskennallisuus menetelmänä ja laskenta teoreettisena lähtökohtana. Jaottelu ei kuitenkään ole ongelmaton: esimerkiksi nykyisin tilastollisten menetelmien kohdalla käytetään poikkeuksetta laskennallista välineistöä. Toisaalta, sosiaalisen verkostojen analyysin kohdalla taustalla on graafiteoria ja sen sovellukset, vaikkakin niitä suoritetaan laskennallisesti. Tässä työssä on syytä esittää tarkempi rajaus laskennallisen yhteiskuntatieteen toimintakentästä, koska rajaus vaikuttaa käsiteltäviin aiheisiin.

\subsection{Kirjallisuuskatsaus}

Havaitaksemme kuinka erilaisia laskennallisten menetelmien käytetään yhteiskuntatieteissä suoritan kirjallisuuskatsauksen laskennallisten mentelmien käytöstä politiikan tutkimuksessa. Etsin kirjallisuuskatsauksessa termejä ``computational social science``, ``machine learning``, `ìnformation extraction``, ``simulation` sekä ``data-mining` artikkeleista käyttäen kunkin lehden omaa hakukonetta, lehdet olivat American Journal of Political Science  (AJPS) sekä European Journal of Political Research (EJPR). %% Political Studies (PS) sekä Political and Society (P\&S).
Lisäksi tarkastellaan Social Science Computer Review:tä (SSCR) merkittävänä laskennallisten yhteiskuntatieteiden lehtenä, vaikkei kyseessä ole erityisesti politiikan tutkimuksen lehti, näihin hakuihin liitettiin täsmentävä termi ``politics``, jolloin kirjallisuuskataus rajataan politiikan tutkimukseen.

Löydettyjen artikkelien määrät on esitetty taulukossa \ref{kirjallisuuskataus}, tarkempaan tarkasteluun valittiin kustakin lehdestä kullakin hakusanalla 20 artikkelia, jotka hakukoneen perusteella vastasivat parhaiten hakutermejä, kustakin artikkelista tarkastettiin tarkemmin metodologia-osuudesta käytettyjä menetelmiä tarkemmin. Artikkelit jaettiin käytettyjen menetelmien osalta agenttipohjaisiin simulointeihin, topic modelleihin ja muihin menetelmiin aineiston luokitteluksi sekä piilotettuihin Markovin ketjuihin (\textit{hidden Markov chain}). Lisäksi osassa artikkeleita ei käytetty laskennallisia menetelmiä tai laskennallisia menetelmiä käytettiin tilastollisten menetelmien testaukseen sekä tarkastamiseen, kuten Monte Carlo-simulaatiot tai bootstrap-tyyliset testit.

\begin{table}

\begin{tabular}{lrrrrr}
~  & Computational Social Science & Machine learning & Information extraction & Simulation & Data-Mining \\
\hline
AJPS  & 53  & 38  & 0  & 162 & 68  \\
EJPR  & 17  & 47  & 0  & 87  & 147 \\
%% PS    & 15  & 243 & 41 & 86  & 172 \\
%% P\&S  & 4   & 63  & 0  & 12  & 76  \\
SSCR & 149 & 228 & 31 & 326 & 36  \\
\hline
\end{tabular}
\caption{Hakuosumat kirjallisuudesta}
\label{kirjallisuuskataus}

\end{table}


\begin{table}

\begin{tabular}{lc}
~  & Computational Social Science \\
\hline
Agenttipohjainen simulijnti & 2 \\
Topic model ja muu tekstin luokittelu & 3 \\
Hidden Markov Models & 1 \\
\hline
Ei selkeää laskennallista menetelmää & 7 \\
Tilastollisten testien tukena &  7 \\
\hline
\end{tabular}
\caption{Tarkempi menetelmällinen valikoima}
\label{kirjallisuuskataus2}

\end{table}


%% On myös syytä perustella tutkimusaiheen relevanttius sekä yhteiskuntatieteellisen että teknillistieteellisen näkökulman kautta. Eräs syy laskennallisten menetelmien yleistymiseen on laskennan resurssien saavutettavuus \cite{z,x}, minkä johdosta erilaisia menetelmiä nostetaan esille usein. Uudet laajat tietovarastot, niin

\section{Simulointi agenttipohjaisilla malleilla}

Yhteiskuntatieteissä simulaatiopohjaista mallinnusta on sovellettu tarkastellessa yhteisöjen toimintaa ja niiden rakentumista \citep[esimerkiksi][]{Epstein2002a,Epstein2001,Cederman2003,Villatoro2013} kuin organisaation toimintaa \citep[esimerkiksi][]{VonRandow2011,Pearson2011}. Perustana on ajatus, että yksittäisten toimijoiden päätökset vaikuttavat kokonaisuutena yhteiskunnan rakenteisiin -- erityisesti tarkastelemalla joukon heterogeenisyyttä sekä sosiaalisen vuorovaikutuksen merkitystä \citep{Squazzoni2013}.

\citet{Epstein2002a} mallintaa kansalaistottelemattomuuden muutosta poliisien ja aktivistien suhteen. Aktivistit reagoivat legimiteetin ja kokemansa vaikeuden (\textit{hardship}) muutokseen: vähäisempi legitimiteetti ja suurempi koettu vaikeus johtavat radikalisoitumiseen. Lisäksi mallissa muutetaan agenttien liikkumista, poliisien määrää ja poliisien vangitsemien aktivistien vankilassa pitämää aikaa. Eri arvojen pohjalta suoritetaan kahdeksan ajoa, joiden pohjalta tarkastellaan yhteiskunnan tilaa. Erilaisten ajojen määrä tässä julkaisussa on siis varsin vähäinen, mikä osittain kyseenalaistaa havaittuja tuloksia -- jotka kuitenkin perustuvat sattumaan\footnote{Esimerkiksi \citet{Dragicevic2014} esittelevät, kuinka tilastollisesti merkittävät erot voivat aiheutua pienistä muutoksista simulaatiossa.}. Lisäksi mallin pohjalta tarkastellaan esimerkiksi legitimiteetin ja poliisien määrän muutosten vaikutusta yhteiskuntaan.

Edelleen \citet{Epstein2001} arvio sosiaalisten normien kehittymistä määrittelemällä kullekkin agentille ensisijaisen normin sekä tarkkailuetäisyys, jonka perusteella agentti mukauttaa omaa normiaan muiden normeihin. Kyseessä siis on sosiaalisoitumisprosessin simulointi laskennallisesti. Jälleen mallien parametreja muutetaan, jotta voidaan havaita erilaisia yhteiskunnallisia tilanteita ja sääntöjen vaikutusta lopputuloksiin; kutakin parametria kuitenkin simuloidaan vain kerran. Myös \cite{Villatoro2013} paneutuvat normien toimintaan simuloinnissan, tarkkailen rankaisun merkitystä normien syntymisessä ja ylläpitämisessä. Työn voi myös tulkita edustavan modernimpaa simulaatiotutkimusta: simulaatiota on toistettu 5000 kertaa tulosten vahvistamiseksi, ja simulaatioaineiston saatavuus on erikseen mainittu artikkelissa.

Yhteiskuntatieteellisten teorioiden lisäksi agenttipohjaisen simulaation sovelluskohteena on ollut erityisesti julkisten organisaatioiden toiminta ja sen kehittäminen. Esimerkiksi \citet{Pearson2011,VonRandow2011} arvioivat vanhustenhoidon tulevia tarpeita simulaatiomallin pohjalta. Kyseessä on mikrosimulaatiomalli, jolloin mallin taustaoletukset perustuvat saatavissa oleviin tietoihin, toisin kuin agenttipohjaisissa malleissa -- missä parametrien arvojen valinta on ollut mielivaltaista.

%% Yllä käsiteltiin sosiaalisen prosessin tutkimusta, eräs varhainen -- ei laskennallinen vaan behavioralistiseen koulukuntaan -- kuuluva tutkimus on  \citet{Granovetter1978} yhteisön toimintaa ja sen käynnistymistä käsittelevä tutkimus.

\subsection{Katsaus teoriaan}

\begin{figure}
\includegraphics[height=6cm]{images/agenttimalli.png} 
\caption{Agenttipohjainen simulaatio}
\cite{Macal2009} mukaan.
\end{figure}

Agenttipohjaisesissa malleissa (\textit{agent-based models, ABM}) mallinnetaan yhteisön toimintaa yksittäisten toimijoiden kautta. Kukin toimija, eli agentti, on yksilöllinen toimintaheuristiikka, tai prefrenssifunktio. Tämän preferenssifunktion sekä mahdollisen satunnaisfuntion avulla määritellään agentin tila seuraavalla ajanhetkellä: \begin{equation}\mathcal{T}_{i,(t+1)} = \mathcal{P}( \mathcal{T}_{i,t} ) + \varepsilon \label{abm:yhtälö}\end{equation} Toistamalla sama simulaatio järjestelmän kaikille agenteille, voidaan määrittää järjestelmän tila ajanhetkellä $t+1$ ajanhetken $t$ perusteella. Luonnollisesti preferenssifuntio $\mathcal{P}$ voi ottaa huomioon myös muiden agenttien toimeinpiteet ja tästä syntyneen järjestelmän tilan tehdessään päätöstä tulevasta tilasta \citep[esimerkiksi][]{Bonabeau2002}. \todo{toinen lähde?} Mallinnusprosessia siis joudutaan etukäteen määrittelemään merkittäviä muuttujia, joka \citet{Gilbert1993} mukaan helpottaa teorioiden muodostamista ja testaamista: simuloidussa malleissa kun ei voida ottaa huomioon kaikki yhteiskunnallista ilmiötä kuvaavia muuttujia.

\citet{Bonabeau2002} kuvaa tarkemmin agenttipohjaisen mallien hyötyjä. Hänen mukaansa agenttipohjainen mallinnus havaitsee paremmin nousevia ilmiöitä {\textit{emergent phenomena}), eli ilmiöitä joita ei voida havaita tarkastellessa vain yksilöitä, vaan yksilöiden välinen vuorovaikutus tai toisten yksilöiden toiminta vaikuttaa yksilön käyttäytymiseen. Esimerkkeinä täläisistä tilanteista on oppiminen ja sopeutuminen (ajallinen korrelaatio), aikaisempi kokemus ja tehdyt päätökset (muisti sekä polkuriippuvuus) sekä epälineaarinen käyttäytyminen, esimerkiksi tietyn kynnysarvon jälkeen tapahtuva toiminta.

Toisena hyötynä \citet{Bonabeau2002} mainitsee agenttipohjaisen mallin helpomman tulkittavuuden. Hän arvioi, että yksilökeskeinen kuvaus toiminnasta on selkeämpi kuin erilaiset tilasiirtymäkuvaukset tai prosessikuvaukset. Tämän takia hän arvioi agenttiohjaisen mallien olevan helpommin asiantuntijoiden tulkittavissa ja arvioitavissa, jolloin mallit ovat selkeämmin validoitavissa. Lisäksi \citet{Bonabeau2002} kritisoi keskisuureiden käyttämistä ilmöiden kuvaamiseksi, koska näissä tilanteissa ääriarvot eivät ole nähtävissä ja keskisuureet tasoittavat vaihtelua.

Agenttipohjaisten mallien lisäksi on mahdollista käyttää mikrosimulaatiota  (\textit{microsimulation model}). Sen toiminta ei merkittävästi eroa esitetystä yhtälöstä \ref{abm:yhtälö}, eli malli pyrkii ennustamaan yksilön toimintaa. Kuitenkin taustalla oleva lähestymistapa eroaa. Agenttipohjaisen malli pyrkii luomaan ilmiön parametrien avulla ja tämän jälkeen tarkastelemaan luomiseen tarvittuja parametreja, kun taas mikrosimulaatiossa ilmiöön liittyvät taustamuuttujat ja niiden väliset määritellään etukäteen \citep[58--59]{Gilbert2005}. Yksilökeskeisen mallinnustavan takia pidän mikrosimulaatioita ja agenttipohjaisia malleja samankaltaisina agenttipohjaisina simulaatioina, vaikka niiden toiminnan yksityiskohdat eroavatkin toisistaan.

\subsection{Kirjastot ja kehitysympäristöt}

Agenttipohjaisen simulaation toteutukseen on olemassa useita erilaisia valmiita kirjastoja sekä alustoja. \cite{Tobias2004} esittää arviointikriteereitä onnistuneille simulaatioympäristöille: ideaalisti kehitysympäristö pystyy automaattisesti luomaan agentteja tiettyjen todennäköisyysmallien perusteella, yksittäiset agentit voivat olla monimutkaisia ja pystyvät viestimään toisilleen. Lisäksi kehitysympäristön arvioinnissa tulisi heidän mukaansa ottaa huomioon kehittäjien tarpeita, esimerkiksi asennuksen yksinkertaisuus, graafisen käyttöliittymän käyttö ja kattava dokumentaatio ovat hyvän kehitysympäristön tunnusmerkkejä. Lisäksi \citet{Tobias2004} ehdottavat avoimen lähdekoodin lisenssejä positiivisena tekijänä, koska niiden käyttö mahdollistaa mallin muutokset.

Näiden kriteerien pohjalta \citet{Tobias2004} päätyvät suosittelemaan RePast-ympäristöä. RePast sallii useamman kielen käytön simulaation kehityksessä, mikä on mahdollistanut erilaisten välineiden ja kirjastojen käytön kehityksessä \cite{North2006}, kuten esimerkiksi Javan käytön. Esimerkki Java-pohjaisesta RePast-mallista on liitteessä \ref{appedix:repast}, RePast-ympäristön hyödyn havaitsee valmiista koodista, jotka liittyvät simulaation pohjaan, siinä tehtävään etsintään ja liikkumiseen sekä todennäköisyysjakaumien käsittelyyn. Lisäksi merkitsemällä (\textit{annotate}), voidaan erikseen merkitä kuinka usein kyseinen agentti suorittaa oman preferenssifunktionsa.

\todo{Toinen framework?}


\section{Luokittelu ohjaamattomalla koneoppimisella}

\citet{Jurek2013} soveltavat koneoppimista 271 valtion demokratian tilan tarkastelussa. He nostavat esille, että koneoppimisen soveltaminen mahdollistaa selittävien tekijöiden muutoksien valitsemisen vapaammin kuin perinteisten tilastollisten mentelmien soveltaminen. Heidän mallinsa löysi yhteensä 210 sääntöä, jotka vaikuttavat demokratian tilaan.

%% \citet{Leskovec2010} regressio-classifier

\cite{collingwood12}


%% Kaufman and Rousseeuw (1990)

\subsection{Menetelmät}

Ohjaamatonta koneoppimista voi lähestyä useilla erilaisilla menetelmillä, kuten assosiaatiosäännöillä, dimensioiden vähentämisellä (\textit{dimensionality reduction}) sekä  klusteroimalla \citet[485--586]{Hastie2009}\footnote{\citet[43--138]{Hastie2009} nostavat myös regressiomallit koneoppimisen osana, mutta kuten johdannossa argumentoin kyseessä on yhteiskuntatieteen näkökulmasta enemmänkin perinteinen laskennallinen menetelmä, enkä siksi esittele tarkemmin näiden yksinkertaisten mallien käyttöä osana laskennallista yhteiskuntatiedettä.}. \todo{Lisää johdattelua?}

Assosisaatiosäännöt perustuvat aineiston ominaisuuksien käsittelyyn siten, että ne selittävät mahdollisimman suuren osan aineistossa havaituista eroista. Kustakin säännöstä lasketaan erikseen sen selitysaste, mutta myös luottamus kyseisen säännön yleistettävyyteen ja säännön nosto (\textit{lift}), jolla arvioidaan löydetyn säännön merkittävyyttä \citep[485--586]{Hastie2009}. Eräs algoritmi sääntöjen muodostamiseksi on \citet{Agrawal1994a} esittelemä Apriori. Apriori laskee eri sääntökombinaatioiden frekvenssin aineistosta, ja tällä perusteella määrittelee mitkä säännöt ovat tarpeeksi yleisiä yleistämiseksi. \todo{pseudoalgoritmin esittely? pitäisi linkata jotenkin laskennan tulos luottamukseen ja lifteihin}

Dimensioiden vähentämisellä viitataan prosessiin, jossa ilmiötä kuvaavien muuttujien määrää vähennetään etsimällä muuttujaryhmiä, joiden kautta aineiston vaihtelua selitetään mahdollisimman hyvin. Yhteiskuntatietelijöiden yleisesti tuntema menetelmä tähän on pääkomponenttianalyysi (\textit{principal component analysysi, PCA}), jonka kautta aineisto jaetaan selittäviin faktoreihin. \todo{Lisää PCAsta?} Laskennalliset menetelmät mahdollistavat myös aineiston samankaltaisten alkioiden esittämisen ryhminä, eli klusteroinnin. Menetelmiä klusterointiin on useita, esimerkiksi $k-means$ sekä spektriklusterointi\todo{Spectral Clustering} \citep[485--586]{Hastie2009}.

k-means-menetelmä pyrkii, nimensä mukaisesti, löytämään keskiarvon kullekkin klusterille ja valitsemaan klustereiden sijainnit siten, että havaintojen sijoittaminen näille keskiarvopisteille aiheuttaa mahdollisimman pienen virheen. Kyseessä on iteratiivinen algoritmi, jota toistetaan kunnes klustereiden sijainnit eivät vaihdu. \todo{pseudokoodia?} Vastaavasti spektriklusterointi perustuu samankaltaisten ominaisarvojen ryhmittämistä, jolloin klusterit korostavat samankaltaisten ominaisuuksien joukkoja paremmin kuin keskiarvoon perustuva k-means klusterointi.

\subsection{Kirjastot ja kehitysympäristöt}

\section{Laskennallinen laadullinen tutkimus}

Johdattelu, laadullinen tutkimus yhteiskuntatieteissä

\subsection{Laskennan teoria}

\subsection{Sovellukset yhteiskuntatieteissä}

\section{Keskustelu}

\subsection{Validiteetti ja reliabiliteetti haasteena}

\todo{Teorian puute? Tarkastelu perinteisellä menetelmällä?}

%% Villatoro2013

%% Edmonds, B., & Hales, D. (2005). Computational simulation as theoretical experiment. Journal of Mathemat- ical Sociology, 29, 209–232

%% A collaborative approach to bridging the research-policy gap through the development of policy advice software. Milne, B et al. Evidence and Policy. (2013) http://dx.doi.org/10.1332/174426413X672210

%% Pearson2011

\subsection{Työskentely monitieteellisesti}

Bibliograafinen analyysi

Riski joutua aputieteeksi

\begin{figure}

%\begin{subfigure}{.6\textwidth}
\includegraphics[width=.6\textwidth]{images/monitieteellisyys_ongelma.png} 
\caption{Menetelmäkehityksen ero soveltajaan \citep[mukaillen][7]{Wijk2006}}
\label{fig:domain_expert_vs_computation_specialist}
%\end{subfigure}
%~~
%\begin{subfigure}{.2\textwidth}
%\includegraphics[width=\textwidth]{images/monitieteellisyys_ongelma.png} 
%\caption{Menetelmäkehityksen ero soveltajaan \citep[mukaillen][7]{Wijk2006}}
%\end{subfigure}

\end{figure}

\citet{Wijk2006} käsittelee monitieteellistä yhteistyötä visualisoinnissa. Oman kokemuksensa pohjalta hän huomauttaa, että alan asiantuntijalla (\textit{domain expert}) ja visualisaation tutkijalla on usein erilaiset tavoitteet visualisaation kehittämisessä: kun visulisaation tutkija ensisijaisesti kehittää uusia visualisaatiomenetelmiä, niin alan asiantuntijan tavoitteena on hyödyllisen työkalun kehittäminen -- mikä voidaan saavuttaa myös perinteisillä menetelmillä. Kuvassa \ref{fig:domain_expert_vs_computation_specialist} esitän saman erottelun sovellettuna laskennalliseen yhteiskuntatieteeseen, alan asiantuntijan ja tietojenkäsittelytietelijöiden välisenä tarkasteluna.

\citet{Wijk2006} jatkaa, että tavoite-eron takia yhteistyön muoto on joku seuraavista: tietojenkäsittelytieteen asiantuntija voi hänen mukaansa toimia työvälinekehittäjänä tai jatkaa tietojenkäsittelyn menetelmien kehittämistä. Tietojenkäsittelytieteen asiantuntija voi myös soveltaa käyttäjäkeskeisiä menetelmiä kehitystyössään, jolloin laskennallisia menetelmiä kehitetään yhteistyössä alan asiantuntijan kanssa. Toisaalta, tietojenkäsittelyttieteen asiantuntija voi myös itse tutustua alaan tai kehittää visualisaatiotekniikoita aiheisiin, joista hän on kiinnostunut. van Wijk kutsuu viimeistä yhteistyön muotoa mielenkiintovetoiseksi kehitystavaksi (\textit{curiosity driven}), ja nostaa esille muodon haasteen: sellaisenaan tällä toimintatavalla ei voida ratkaista aihealueen haastavimpia ongelmia. Näin laskennallisen yhteiskuntatieteen mahdollisiksi... \todo{johtopäätös: yhteiskuntatieteessä voidaan tehdä UCD tai toolsmith}

\subsection{Paradigman muutos?}

Kuten \citet[265]{watts11} esittääkin

\begin{quote}
\texttt{[o]}n kuitenkin välttämätöntä soveltaa kaikkia näitä \texttt{[sekä laskennallisia että kuvailevia]} lähetymistapoja samanaikaisesti, pyrkien saavuttamaan johtopäätöksiä siitä, kuinka ihmiset käyttäytyvät ja kuinka maailma toimii -- sekä ylhäältä että alhaalta, käyttäen hyödyksi kaikkia menetelmiä jotka ovat käytettävissä. (\textit{oma suomennus})
\end{quote}

Laskennallisille menetelmille yhteiskuntatieteissä on siis tarvetta, koska niiden avulla on mahdollista esittää ratkaisuja uusiin kysymyksiin. Kuitenkin, samaan aikaan tulee huomioida yhteiskuntatieteiden oppihistoria monimenetelmäisenä ja -paradigmaisena: eri lähestymistapojen käyttö samojen ongelmien käsittelyyn on perinteinen menetelmä yhteiskuntatieteiden osalta. Tämän takia monitieteelliinen työskentely on tarpeen, mutta valitettavasti tällä hetkellä laskennallinen yhteiskuntatiede ei yleisesti keskustele 

\section{Johtopäätökset (2)}

Työssä käsiteltiin laskennallisia menetelmiä yhteiskuntatieteiden toiminnassa, ja tätä kautta muodostunutta laskennallisen yhteiskuntatieteen toimikenttää. Työn ensimmäinen kontribuutio käsittelee toimintakentän määritelmää: tässä työssä käytettiin varsin rajaavaa määritelmää laskennallisesta yhteiskuntatieteestä (katso sivu \pageref{css-maar}), missä korostettiin laajojen tietomassojen käyttöä ennustavien mallien luomiseksi. Kuitenkin, laskennallinen yhteiskuntatiede voidaan nähdä laajemmin, esimerkiksi \citet{cioffi-revilla10} esittää huomattavasti tätä laajemman määritelmän, ja halutessa esimerkiksi moderni tilastotiede täyttää monia laskennallisuudelle asetettuja oletuksia: elaboroi. Tämän takia dikotomisen asettelun sijaan on mielekästä puhua laskennallisen yhteiskuntatieteen jatkumosta. elaboroi.

Toinen merkittävä kontribuutio pohtii laskennallisen yhteiskuntatieteen asemaa alana. Korostin tarvetta poikkitieteelliseen työskentelyyn, missä sekä laskennallisten menetelmien asiantuntijat että kohdealueen asiantuntijat työskentelisivät yhdessä. Lisää. Arvioin myös, muodostaako laskennallinen lähestymistapa paradigman muutoksen yhteiskuntatieteissä... Kuitenkin, ilmeistä on, että laskennallisuus on uusi väline, joka on käytettävissä yhteiskuntatieteessä. lisää...

Lisäksi työssä on esitelty kolmea laskennallista menetelmää tarkemmin ...

% \section*{Kiitokset}

% Kiitän työn ohjaajaa, Antti Ukkosta, keskusteluista ja kyseenalaistavista kysymyksistä. Lisäksi kiitän Airi Lampista sekä Eric Malmia työn aikana käydyistä keskusteluista sekä \texttt{pamee}ta irkkiavautumisien kuuntelusta.


\newpage
\bibliographystyle{abbrvnat}
\bibliography{bsc,/Users/mnelimar/Dropbox/_tools/library.bib} 

\newpage

\appendix

\section{RePast-esimerkki}
\label{appedix:repast}

\lstinputlisting[language=Java,breaklines=true,numbers=left,tabsize=1]{examples/RePast.java} 

RePast-dokumentaation pohjalta luotu esimerkki.

%% http://repast.sourceforge.net/docs/RepastJavaGettingStarted.pdf

\end{document}

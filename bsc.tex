\documentclass[finnish,gradu,twoside,12pt]{tktltiki}

\usepackage[utf8]{inputenc}
\usepackage{natbib}


\usepackage{graphicx}
\usepackage{subfigure}
\usepackage{hyperref}
\usepackage{rotating}
\usepackage{float}
\usepackage{microtype}

\begin{document}

\onehalfspacing
%\doublespacing

\title{Laskennallinen yhteiskuntatiede}
\author{Matti Nelimarkka}
\date{5.5.2011}
\level{LuK --tutkielma}

\numberofpagesinformation{}

\keywords{}

\maketitle

\begin{abstract}
Työssä esitellään laskennallisten menetelmien vaikutusta yhteiskuntatieteelliseen tutkimukseen kolmantena laajana menetelmällisenä perheenä laadullisen ja määrällisen tutkimuksen lisäksi. Laskennallinen yhteiskuntatiede (\textit{computational social science}) soveltaa moderneja laskennallisia menetelmiä, kuten kone-oppimista ja tilastollista analyysiä, yhteiskuntaa koskevien kysymysten esittämiseen ja tulkintaan.

Erityisesti työssä keskitytään n menetelmään, lista. Menetelmät on valittu esittämään tällä hetkellä käytössä olevia laskennallisen yhteiskuntatieteen sovelluskohteita. Kunkin menetemän kohdalta esitetään lyhyesti sen laskennallinen teoria sekä tutkimuksia, missä kyseistä menetelmää on sovellettu.

Yksittäisten tutkimusten tarkastelun kautta voidaan nostaa esille suurempia teemakokonaisuuksia laskennallisen yhteiskuntatieteen kehityksen kannalta. Ensimmäisenä käsittelen laskennallisen yhteiskuntatieteen validiteettiä ja reliabiliteettia, eli tulosten oikeellisuutta ja uskottavuutta. Toinen merkittävä teema on monitieteellinen työskentely, minkä koen välttämättömänä laskennallisen yhteiskuntatieteen alalla, mutta tunnustan siihen liittyvät ongelmat. Viimeisenä keskustelun osiona pohdin, voiko laskennallinen tiede edustaa kuhnilaisessa mielessä paradigman muutosta, millä olisi laajempi vaikutus yhteiskuntatieteen kehitykseen ja tulevaisuuteen.
\end{abstract}

\mytableofcontents

\section{Johdanto}

Yhteiskuntatiede toimintaympäristönä

Eräs motivatio laskennallisten menetelmien kehittymiseen on ollut Internetin tarjoamat laajat tietovarannot ihmisten käyttäytymisestä \citep{}. Nämä aineistot ovat olleet niin laajoja, että ne tarjoavat uusia mahdollisuuksia aikaisemmin aineiston keräämisen monimutkaisuuden takia vähemmän käytettyihin menetelmin, kuten esimerkiksi verkosto-analyysiin. Toisaalta, Internetin laajojat aineistot ovat myös nostaneet esille myös mahdollisuuksia perinteisten yhteiskunnallisten teorioiden testaamiseen näissä ympäristöissä.

Samaan aikaan on syytä huomioida, että vaikka Internet onkin yksi uusi lähde aineistoille, se ei ole ainoa. Esimerkiksi aineistoa on aikaisemmin kerätty ihmisten elinympäristöstä \citep{eagle06,oulasvirta12}, ja nykyinen avoimen datan liikkeen (\textit{open data}) avulla myös muita laajoja tietovarantoja voidaan avata yhteiskuntatieteen käyttöön.

Laskannallisen yhteiskuntatieteen määritelmä

\section{Menetelmä 1}

\section{Menetelmä n}

\section{Keskustelu}

\subsection{Validiteetti ja reliabiliteetti haasteena}

\subsection{Työskentely monitieteellisesti}

\subsection{Paradigman muutos?}

\section{Johtopäätökset}

\bibliographystyle{abbrvnat}
\bibliography{bsc}  

\end{document}

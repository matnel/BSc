\documentclass[finnish,gradu,twoside,12pt]{tktltiki}

\usepackage[utf8]{inputenc}
\usepackage{natbib}


\usepackage{graphicx}
\usepackage{subfigure}
\usepackage{hyperref}
\usepackage{rotating}
\usepackage{float}
\usepackage{microtype}

\begin{document}

%% \onehalfspacing
%\doublespacing

\title{Laskennallinen yhteiskuntatiede}
\author{Matti Nelimarkka}
\date{5.5.2011}
\level{LuK --tutkielma}

\numberofpagesinformation{}

\keywords{}

\maketitle

\begin{abstract}
Työssä esitellään laskennallisten menetelmien vaikutusta yhteiskuntatieteelliseen tutkimukseen kolmantena laajana menetelmällisenä perheenä laadullisen ja määrällisen tutkimuksen lisäksi. Laskennallinen yhteiskuntatiede (\textit{computational social science}) soveltaa moderneja laskennallisia menetelmiä, kuten kone-oppimista ja tilastollista analyysiä, yhteiskuntaa koskevien kysymysten esittämiseen ja tulkintaan.

Erityisesti työssä keskitytään n menetelmään, lista. Menetelmät on valittu esittämään tällä hetkellä käytössä olevia laskennallisen yhteiskuntatieteen sovelluskohteita. Kunkin menetemän kohdalta esitetään lyhyesti sen laskennallinen teoria sekä tutkimuksia, missä kyseistä menetelmää on sovellettu.

Yksittäisten tutkimusten tarkastelun kautta voidaan nostaa esille suurempia teemakokonaisuuksia laskennallisen yhteiskuntatieteen kehityksen kannalta. Ensimmäisenä käsittelen laskennallisen yhteiskuntatieteen validiteettiä ja reliabiliteettia, eli tulosten oikeellisuutta ja uskottavuutta. Toinen merkittävä teema on monitieteellinen työskentely, minkä koen välttämättömänä laskennallisen yhteiskuntatieteen alalla, mutta tunnustan siihen liittyvät ongelmat. Viimeisenä keskustelun osiona pohdin, voiko laskennallinen tiede edustaa kuhnilaisessa mielessä paradigman muutosta, millä olisi laajempi vaikutus yhteiskuntatieteen kehitykseen ja tulevaisuuteen.
\end{abstract}

\mytableofcontents

\section{Johdanto (3)}

Työssä esitetään laskennallisia menetelmiä joita sovelletaan yhteiskuntaa koskeviin kysymyksiin vastaamiseksi: laskennallisen yhteiskuntatieteen malleihin, sovelluksiin sekä tieteenfilosofiaan. Työssä yhteiskuntatiede nähdään laajasti yhteiskuntaa tutkivien tieteiden joukkona, joihin kuuluvat esimerkiksi politiikan tutkimus, taloustiede, sosiologia, sosiaalipyskologia ja viestinnän tutkimus\footnote{Tässä työssä esimerkit on erityisesti valittu politiikan tukimuksen näkökulmasta, mutta esitetyt menetelmät ovat laajemmin sovellettavissa laajemminkin yhteiskuntatieteisiin.}. Tutkimusaloille tyypillistä on samankaltaisten tutkimuskysymysten lähestyminen kunkin alan oman oppihistorian sekä käsitteistön kautta \citep{a}. Esimerkiksi poliitikkojen sosiaalisen median käyttöä voidaan lähestyä niin politiikan tutkimuksen, viestinnän tutkimuksen kuin esimerkiksi sosiologian näkökulmasta. Esimerkki.

Kuten yllä olevista esimerkeistä huomataan, eri yhteiskuntatieteet käsittelevät samankaltaisia kysymyksiä, kuitenkin erilaisin painotuksin. Nykyinen eriytyminen eri tieteenaloihin on osa tieteenalan laajempaa kehittymistä nykyiseen käsitykseen yhteiskuntatieteestä. Jotta yhteiskuntatieteen nykyinen tila, sen puutteet ja laskennallisten menetelmien tarjoamat mahdollisuudet ymmärrettäisiin, niin on tarpeen esitellä yhteiskuntatieteen oppihistoriaa tiiviisti. Eräs merkittävä muutos joka valtavirran tutkimuksessa sijoittuu 1950 -- 1960-luvuille on behavioralismin vaikutus yhteiskuntatieteisiin. Ennen tätä muutosta yhteiskuntatieteet, varsinkin 1800-luvun lopussa sekä 1900-luvun alussa, olivat kuvailevia ja normatiivisia: erilaisia havaintoja kuvailtiin varsin laajasti \citep{x} ja yhteiskuntafilosofia\footnote{Jonka historia toki on merkittävästi laajempi, esimerkiksi Platonin Valtio voidaan nähdä yhteiskuntaa kuvaavana filosofiana ja Hobbesin sekä Locken sopimusteoriamallien kautta muodostetut käsitteellistykset valtiosta toimijana ovat hyviä esimerkkejä perinteisestä normatiivisesta ajattelusta} oli yhteiskuntatieteen keskeistä sisältöä. Kuitenkin, behavioralismin myötä käänne empiiriseen tutkimukseen, todellisten ilmiöiden havainnointiin ja niiden arviointiin, oli merkittävä. Tätä siirtymää kuvaa \citep[yy]{x} kuvaus yhteiskuntatieteestä:

TODO

Samaan aikaan behavioralismi kuitenkin loi odotuksia käytetyistä lähestymistavoista, behavioralistit kokivat, että luonnontieteiden lähestymistapaa tulisi soveltaa yhteiskuntatieteissä. Tämä nosti esille vastaliikkeen, kriittisen koulukunnan. Kriittisen koulukunnan mukaan yhteiskunnan tutkimuksessa on tarpeen myös luonnontieteistä poikkeava käsittelytapa, missä esimerkiksi tutkimushypoteesien rooli ei ole yhtä keskeinen. Samoin kriittinen koulukunta kyseenalaisti behavioralismin objektiivisuuden, argumentoiden, että tutkimuksen operationalisoinnit luovat jo subjektiivisen ulottuvuuden ja tulosten tulkinta on edelleen subjektiivista. Tietenkin, myös kriittisen koulukunnan työskentelyä kuvaa subjektiivisuuden keskeinen luonne tutkimuksen osana -- mutta, tämä subjektiivisuus onkin usein osa kaikkea yhteiskuntatiedettä.

Modernissa empiirisessä yhteiskuntatieteesä voidaan siis nostaa esille kahden laajemman lähestymistavan soveltaminen tutkimuskysymyksiin. Tutkimuksessa sovelletaan yleensä laadullista tai määrällistä tutkimusotetta, missä ensimmäisessä pääpaino on enemmän kuvailevassa aineistossa ja sen tulkinnassa, jälkimmäinen taas pyrkii mitattaviin aineistoihin ja selittämiseen matemaattisia menetelmiä hyödyntäen \citep{a,b}. Sellaisenaan voimakasta menetelmäjakoa on kritisoitu, koska useiden ilmiöiden kohdalla olisi mielekästä soveltaa molempia tutkimusmenetelmiä: laadullisen tutkimuksen perinteinen haaste on tutkimuksen yleistettävyys koko joukkoon, kun taas määrällisen tutkimuksen tapauksessa kausaalisen tekijän keskeinen luonne voi helposti jäädä selvittämättä \citep{a,b,c}.

Kritiikkini kausaalisuuden puutteesta voi olla yllättävä, joten valaistaan tätä esimerkillä poliitikoiden verkkomedian käytön tutkimuksella. On havaittu, että poliitikot eivät suosi vuorovaikutuksellista ja kaksisuuntaista verkkomedian käyttöä, vaan käyttävät verkkomediaa perinteisen median jatkona \cite{a,b}. Kuitenkaan, nämä määrällistä lähestymistapaa soveltavat tutkimukset eivät täysin pysty selittämään miksi näin on. Tällöin on tarve laadulliselle tutkimukselle, kuten \cite{x} havannointityölle ehdokkainen parissa. Hänen mukaansa taustalla on ennenkaikkea x,y,z. Näin vakuuttavaan yhteiskuntatieteelliseen argumentaatioon vaaditaan toisaalta määrällistä, koko tutkittavan joukon kattavaa havaintoa sekä laadullista työskentelyä, mikä joko selvittää havaintoja -- kuten yllä -- tai nostaa esille uusia tutkimuskysymyksiä.

Määrällisestä lähestymistapaa edustavat tilastollisten menetelmien soveltaminen aineiston käsittelyssä, mutta myös formaalit menetelmät, kuten peliteoria voidaan laskea osaksi tätä lähestymistapaa \citep{a}. Myös tarkastelun kohteena oleva laskennallinen yhteiskuntatiede (\textit{computational social science}) edustaa määrällistä lähestymistapaa. Laskennalliset tieteet määritellään \citep{a,b,c}. Vastaava esitys on myös nähtävissä laskennallisen yhteiskuntatieteen määritelmissä:

\begin{description}
\item[\citet{cioffi-revilla10}] määrittelee laskennallisen yhteiskuntatieteen laajasti seuraavien laskennallisten menetelmien soveltamisena: tiedon uuttamisen menetelmät (\textit{information extraction}), sosiaalisten verkostojen analyysi, paikkatietojärjestelmien käyttämisen ja mallinnuksen sekä simulaation menetelmät. Hän myös arvioi, että tiedon visualisointimenetelmät voivat myöhemmin laajentua käytettäväksi laskennallisen yhteiskuntatieteen menetelmänä. Hän siis näkee, että laskennallista tiedettä voidaan määritellä tiettyjen menetelmien käyttönä.
\item[\citet{lazer09}] taas esittää laskennallisen yhteiskuntatieteen laajojen aineistomäärien keräämisenä ja käsitttelynä määrällisten menetelmien avulla. He nostavat esille, että laskennallinen yhteiskuntatiede on aineistolähtöistä (\textit{data-driven}), minkä tulkitsen tarkoittavan, että perinteinen hypoteeseihin ja niiden tarkasteluun perustuvat tilastolliset menetelmät eivät olisi laskennallista yhteiskuntatiedettä.
\item[\cite{bankes02}] kuvaavat laskennallisia epistemologioita (\textit{computational epistemology}), eli käsityksiä hyvästä tieteestä. Toisaalta he ehdottavat, että laskennallinen yhteiskuntatieteen tarkoitus on luoda selittäviä malleja, joiden avulla yhteisöä voidaan tutkia. Toisaalta, he nostavat esille myös mahdollisuuden käyttää näitä malleja ei vain yhteisön tutkimiseen vaan myös laskennallisten koeasetelmien suorittamiseen. Tässä määritelmässä siis tärkeintä on tutkittavan ilmiön mallinnus laskennallisin menetelmin.
\end{description}

Kuten havaitaan, tutkimusalan määritelmät eivät ole kaikilla tutkijoilla samankaltaisia, johtuen myös erilaista lähestymistavoista laskennallisen yhteiskuntatieteen määritelmään. Kuten \citet{cioffi-revilla10} huomauttaa, laskennallisen tieteen osalta voidaan erottaa laskennallisuus menetelmänä ja laskenta teoreettisena lähtökohtana. Jaottelu ei kuitenkään ole ongelmaton: esimerkiksi nykyisin tilastollisten menetelmien kohdalla käytetään poikkeuksetta laskennallista välineistöä. Toisaalta, sosiaalisen verkostojen analyysin kohdalla taustalla on graafiteoria ja sen sovellukset, vaikkakin niitä suoritetaan laskennallisesti. Tässä työssä on syytä esittää tarkempi rajaus laskennallisen yhteiskuntatieteen toimintakentästä, koska rajaus vaikuttaa käsiteltäviin aiheisiin.

%% TODO color box
\label{css-maar}

Määrittelen tässä työssä laskennallisen yhteiskuntatieteen menetelmiksi, millä voidaan annettujen oletusten perusteella rakentaa ennustuskykyisiä malleja yhteiskuntatieteellisten kysymysten ratkaisemiseksi. ELABOROI. Tämän rajauksen seurauksena olen nostanut esille kolme erilaista laskennallista menetelmää: agenttipohjaisen simulaation, koneoppimisen sekä .

On myös syytä perustella tutkimusaiheen relevanttius sekä yhteiskuntatieteellisen että teknillistieteellisen näkökulman kautta. Eräs syy laskennallisten menetelmien yleistymiseen on laskennan resurssien saavutettavuus \cite{z,x}, minkä johdosta erilaisia menetelmiä nostetaan esille usein. Uudet laajat tietovarastot, niin Internetin \citep{adamic05,notess02} kuin elinympäristöstä kerättyjen sensoriaineistojen \citep{eagle06,oulasvirta12}, yleistyminen ja käyttöönotoot on helpottanut suurten aineistojen keräämistä. Tällöin perinteiset ei-laskennalliset menetelmät eivät välttämättä sovellu tutkimukseen, ja samaan aikaan laajat aineistot mahdollistavat aiemmin haastavina pidettyjen menetelmien käytön -- esimerkiksi verkosto-analyysin ongelma on ollut aineiston vaikea keräys \citep{a}. Tällöin myös rahoittajien korostama poikkitieteellinen lähestymistapa \citep[60]{tieteentila12} on ensinnäkin mahdollinen resurssien myötä ja toisaalta välttämätön, koska perinteiset lähestymistavat eivät käytä hyödykseen mahdollisuuksia joita laskennallisten tieteiden kehitys on luonut.

\section{Agenttipohjainen simulaatio (3.5)}

\subsection{Laskennan teoria}

\subsection{Sovellukset yhteiskuntatieteissä}

\section{Koneoppiminen (3.5)}

\subsection{Laskennan teoria}

\subsection{Sovellukset yhteiskuntatieteissä}

\cite{collingwood12}

\section{Menetelmä $x$ (3.5)}

\subsection{Laskennan teoria}

\subsection{Sovellukset yhteiskuntatieteissä}

\section{Keskustelu (5)}

\subsection{Validiteetti ja reliabiliteetti haasteena}

\subsection{Työskentely monitieteellisesti}

Bibliograafinen analyysi

Riski joutua aputieteeksi

\subsection{Paradigman muutos?}

Kuten \citet[265]{watts11} esittääkin

\begin{quote}
\texttt{[o]}n kuitenkin välttämätöntä soveltaa kaikkia näitä \texttt{[sekä laskennallisia että kuvailevia]} lähetymistapoja samanaikaisesti, pyrkien saavuttamaan johtopäätöksiä siitä, kuinka ihmiset käyttäytyvät ja kuinka maailma toimii -- sekä ylhäältä että alhaalta, käyttäen hyödyksi kaikkia menetelmiä jotka ovat käytettävissä. (\textit{oma suomennus})
\end{quote}

Laskennallisille menetelmille yhteiskuntatieteissä on siis tarvetta, koska niiden avulla on mahdollista esittää ratkaisuja uusiin kysymyksiin. Kuitenkin, samaan aikaan tulee huomioida yhteiskuntatieteiden oppihistoria monimenetelmäisenä ja -paradigmaisena: eri lähestymistapojen käyttö samojen ongelmien käsittelyyn on perinteinen menetelmä yhteiskuntatieteiden osalta. Tämän takia monitieteelliinen työskentely on tarpeen, mutta valitettavasti tällä hetkellä laskennallinen yhteiskuntatiede ei yleisesti keskustele 

\section{Johtopäätökset (2)}

Työssä käsiteltiin laskennallisia menetelmiä yhteiskuntatieteiden toiminnassa, ja tätä kautta muodostunutta laskennallisen yhteiskuntatieteen toimikenttää. Työn ensimmäinen kontribuutio käsittelee toimintakentän määritelmää: tässä työssä käytettiin varsin rajaavaa määritelmää laskennallisesta yhteiskuntatieteestä (katso sivu \pageref{css-maar}), missä korostettiin laajojen tietomassojen käyttöä ennustavien mallien luomiseksi. Kuitenkin, laskennallinen yhteiskuntatiede voidaan nähdä laajemmin, esimerkiksi \citet{cioffi-revilla10} esittää huomattavasti tätä laajemman määritelmän, ja halutessa esimerkiksi moderni tilastotiede täyttää monia laskennallisuudelle asetettuja oletuksia: elaboroi. Tämän takia dikotomisen asettelun sijaan on mielekästä puhua laskennallisen yhteiskuntatieteen jatkumosta. elaboroi.

Toinen merkittävä kontribuutio pohtii laskennallisen yhteiskuntatieteen asemaa alana. Korostin tarvetta poikkitieteelliseen työskentelyyn, missä sekä laskennallisten menetelmien asiantuntijat että kohdealueen asiantuntijat työskentelisivät yhdessä. Lisää. Arvioin myös, muodostaako laskennallinen lähestymistapa paradigman muutoksen yhteiskuntatieteissä... Kuitenkin, ilmeistä on, että laskennallisuus on uusi väline, joka on käytettävissä yhteiskuntatieteessä. lisää...

Lisäksi työssä on esitelty kolmea laskennallista menetelmää tarkemmin ...

% \section*{Kiitokset}

% Kiitän työn ohjaajaa, Antti Ukkosta, keskusteluista ja kyseenalaistavista kysymyksistä. Lisäksi kiitän Airi Lampista sekä Eric Malmia työn aikana käydyistä keskusteluista sekä \texttt{pamee}ta irkkiavautumisien kuuntelusta.


\newpage
\bibliographystyle{abbrvnat}
\bibliography{bsc}  

\end{document}

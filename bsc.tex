\documentclass[finnish,gradu,twoside]{tktltiki}

\usepackage[utf8]{inputenc}
\usepackage{harvard}
\usepackage{graphicx}
\usepackage{subfigure}
\usepackage{hyperref}
\usepackage{rotating}
\usepackage{float}
\citationmode{abbr}
\usepackage{microtype}

\begin{document}

\onehalfspacing
%\doublespacing

\title{Laskennallinen yhteiskuntatiede}
\author{Matti Nelimarkka}
\date{5.5.2011}
\level{LuK --tutkielma}

\numberofpagesinformation{}

\keywords{}

\maketitle

\begin{abstract}
Työssä esitellään laskennallisten menetelmien vaikutusta yhteiskuntatieteelliseen tutkimukseen kolmantena laajana menetelmällisenä perheenä laadullisen ja määrällisen tutkimuksen lisäksi. Laskennallinen yhteiskuntatiede (\textit{computational social science}) soveltaa moderneja laskennallisia menetelmiä, kuten kone-oppimista ja tilastollista analyysiä, yhteiskuntaa koskevien kysymysten esittämiseen ja tulkintaan.

Erityisesti työssä keskitytään n menetelmään, lista. Menetelmät on valittu esittämään tällä hetkellä käytössä olevia laskennallisen yhteiskuntatieteen sovelluskohteita. Kunkin menetemän kohdalta esitetään lyhyesti sen laskennallinen teoria sekä tutkimuksia, missä kyseistä menetelmää on sovellettu.

Yksittäisten tutkimusten tarkastelun kautta voidaan nostaa esille suurempia teemakokonaisuuksia laskennallisen yhteiskuntatieteen kehityksen kannalta. Ensimmäisenä käsittelen laskennallisen yhteiskuntatieteen validiteettiä ja reliabiliteettia, eli tulosten oikeellisuutta ja uskottavuutta. Toinen merkittävä teema on monitieteellinen työskentely, minkä koen välttämättömänä laskennallisen yhteiskuntatieteen alalla, mutta tunnustan siihen liittyvät ongelmat. Viimeisenä keskustelun osiona pohdin, voiko laskennallinen tiede edustaa kuhnilaisessa mielessä paradigman muutosta, millä olisi laajempi vaikutus yhteiskuntatieteen kehitykseen ja tulevaisuuteen.
\end{abstract}

\mytableofcontents

\section{Johdanto}

\section{Menetelmä 1}

\section{Menetelmä n}

\section{Keskustelu}

\subsection{Validiteetti ja reliabiliteetti haasteena}

\subsection{Työskentely monitieteellisesti}

\subsection{Paradigman muutos?}

\section{Johtopäätökset}

\end{document}
